\documentclass[11pt,DIV=9, letterpaper, oneside, openright]{scrartcl}
\usepackage{graphicx}
\usepackage[utf8]{inputenc}
\usepackage[spanish]{babel}
\title{Edición, mantenimiento y evaluación de una aplicación web para mostrar el fenómeno del suicidio}
\subtitle{Reporte de actividades Abril-2024}
\author{José Adrián Rodríguez González }
\date{Abril 2024}
\begin{document}

\maketitle

\section{Introducción}
Como parte del reporte de actividades, se destacan los siguientes puntos a trabajar, a su vez de continuar con puntos y objetivos pendientes en el anterior mes.

\begin{itemize}
    \item Diseño e implementación de nuevas funcionalidades
    \item Pruebas de despliegue
    \item Preparación de un artículo académico
\end{itemize}

\section{Diseño e implementación de nuevas funcionalidades}

Parte de esta sección se venido realizando desde la corrección de los errores del mes de febrero, marzo y los que quedaron pendientes en marzo.
Uno de los avances más importantes que se ha llevado a cabo fue el despliegue del sitio a manera de tener una previsualización de lo que se ha trabajado.

El siguiente punto fue, que se encontraron fuentes de letra que no se utilizan en el sitio y manera de optimizar el sitio en cuestión de memoria consumida, se eliminaron todos los enlaces  caídos.

El último cambio importante realizado que a su vez quedó pendiente en marzo fue el quitar el menú desplegable cuando se seleccione una ciudad. Ya que cuando se buscaba la ciudad se mantenía la lista de resultados y cuando se usaba en pantallas pequeñas, tapaba gran parte de las personas que se encontraban ilustradas.
Así que se realizó la modificación en el archivo de visualización para ocultar esa ventana la seleccionar una ciudad.

\section{Pruebas de despliegue}

Como parte fundamental de un despliegue de cualquier sitio web, es que sea visualizado en la gran mayoría de dispositivos y navegadores. Las pruebas de visualización se han realizado en los navegadores más comunes, (Microsoft Edge, Google Chrome, Safari y Mozilla Firefox), a lo cuál, han dado buenos resultados en sus despliegues, sin embargo se seguirá probando con más navegadores.

Además, como se mencionó en el reporte del mes pasado se tiene implementada una Github Action, que permite desplegar el sitio en su totalidad a una Github Page. Es importante decir que este proceso es automatizado y depende los PR\footnote{\emph{Pull Request}} que se realizan en el repositorio. 

\section{Preparación de un artículo académico}
Dado ya a la naturaleza del andar del proyecto, se está generando un artículo que describa el proyecto, sus objetivos y sus motivos de realización.
\end{document}